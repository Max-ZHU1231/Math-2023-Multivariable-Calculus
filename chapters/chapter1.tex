\section{Three-Dimensional Space}

\subsection{Rectangular Coordinates in $\mathbb{R}^3$}

We use an ordered triple $(x, y, z)$ to represent a point in the three-dimensional space $\mathbb{R}^3$.
\begin{itemize}
    \item \textbf{Origin:} The point $(0,0,0)$ where the three axes meet.
    \item \textbf{Vector:} An arrow based at one point pointing to another. If $\mathbf{v}$ connects $P_0(x_0, y_0, z_0)$ to $P_1(x_1, y_1, z_1)$, then:
    \[ \mathbf{v} = \langle x_1 - x_0, y_1 - y_0, z_1 - z_0 \rangle = (x_1 - x_0)\mathbf{i} + (y_1 - y_0)\mathbf{j} + (z_1 - z_0)\mathbf{k}. \]
\end{itemize}

\begin{definition}[Vector Operations]
Let $\mathbf{a} = \langle a_1, a_2, a_3 \rangle$, $\mathbf{b} = \langle b_1, b_2, b_3 \rangle$ and $c \in \mathbb{R}$.
\begin{itemize}
    \item \textbf{Addition:} $\mathbf{a} + \mathbf{b} = \langle a_1 + b_1, a_2 + b_2, a_3 + b_3 \rangle$
    \item \textbf{Scalar Multiplication:} $c\mathbf{a} = \langle ca_1, ca_2, ca_3 \rangle$
\end{itemize}
\end{definition}

\begin{property}[Algebraic Properties]
\begin{enumerate}
    \item $\mathbf{a} + \mathbf{b} = \mathbf{b} + \mathbf{a}$ (Commutativity)
    \item $(\mathbf{a} + \mathbf{b}) + \mathbf{c} = \mathbf{a} + (\mathbf{b} + \mathbf{c})$ (Associativity)
    \item $\lambda(\mathbf{a} + \mathbf{b}) = \lambda\mathbf{a} + \lambda\mathbf{b}$ (Distributivity)
\end{enumerate}
\end{property}

\subsection{Dot Product}

\begin{definition}[Dot Product]
The dot product of $\mathbf{a} = \langle a_1, a_2, a_3 \rangle$ and $\mathbf{b} = \langle b_1, b_2, b_3 \rangle$ is the scalar:
\[ \mathbf{a} \cdot \mathbf{b} = a_1 b_1 + a_2 b_2 + a_3 b_3. \]
Note that $\mathbf{a} \cdot \mathbf{a} = |\mathbf{a}|^2$, where $|\mathbf{a}| = \sqrt{a_1^2 + a_2^2 + a_3^2}$ is the norm.
\end{definition}

\begin{theorem}[Geometric Interpretation]
Let $\theta$ be the angle between $\mathbf{a}$ and $\mathbf{b}$. Then:
\[ \mathbf{a} \cdot \mathbf{b} = |\mathbf{a}| |\mathbf{b}| \cos \theta. \]
Consequently, $\theta = \cos^{-1}\left( \frac{\mathbf{a} \cdot \mathbf{b}}{|\mathbf{a}| |\mathbf{b}|} \right)$.
\end{theorem}

\begin{corollary}[Orthogonality]
Two non-zero vectors $\mathbf{a}$ and $\mathbf{b}$ are orthogonal if and only if $\mathbf{a} \cdot \mathbf{b} = 0$.
\end{corollary}

\begin{example}[Example 1.1: Inscribed Triangle]
Show that any triangle inscribed in a circle with one side coinciding with the diameter must be a right-angled triangle.
\end{example}

\begin{solution}
Let $O$ be the center (origin). Let vectors $\mathbf{a}$ and $\mathbf{b}$ represent the radius to the vertices.
The diameter connects endpoints defined by vectors $\mathbf{a}$ and $-\mathbf{a}$. Let $\mathbf{b}$ be the third vertex.
The two sides of the triangle are represented by vectors:
\[ \mathbf{u} = \mathbf{b} - \mathbf{a}, \quad \mathbf{v} = \mathbf{b} - (-\mathbf{a}) = \mathbf{b} + \mathbf{a}. \]
Computing their dot product:
\[ \mathbf{u} \cdot \mathbf{v} = (\mathbf{b} - \mathbf{a}) \cdot (\mathbf{b} + \mathbf{a}) = \mathbf{b} \cdot \mathbf{b} - \mathbf{a} \cdot \mathbf{a} = |\mathbf{b}|^2 - |\mathbf{a}|^2. \]
Since both $\mathbf{a}$ and $\mathbf{b}$ are radii, $|\mathbf{a}| = |\mathbf{b}|$. Thus $\mathbf{u} \cdot \mathbf{v} = 0$, proving they are orthogonal.
\end{solution}

\subsection{Cross Product}

\begin{definition}[Cross Product]
The cross product $\mathbf{a} \times \mathbf{b}$ is a vector defined by:
\begin{itemize}
    \item \textbf{Magnitude:} $|\mathbf{a} \times \mathbf{b}| = |\mathbf{a}| |\mathbf{b}| \sin \theta$ (Area of the parallelogram spanned by $\mathbf{a}, \mathbf{b}$).
    \item \textbf{Direction:} Orthogonal to both $\mathbf{a}$ and $\mathbf{b}$ (Right-hand rule).
\end{itemize}
\end{definition}

\begin{theorem}[Determinant Formula]
\[
\mathbf{a} \times \mathbf{b} =
\begin{vmatrix}
\mathbf{i} & \mathbf{j} & \mathbf{k} \\
a_1 & a_2 & a_3 \\
b_1 & b_2 & b_3
\end{vmatrix}
= (a_2 b_3 - a_3 b_2) \mathbf{i} - (a_1 b_3 - a_3 b_1) \mathbf{j} + (a_1 b_2 - a_2 b_1) \mathbf{k}.
\]
\end{theorem}

\begin{example}[Example 1.2]
Find a vector $\mathbf{n}$ orthogonal to the plane passing through $A(0, 2, -1)$, $B(4, 0, -1)$, and $C(7, -3, 0)$. Also find the area of $\triangle ABC$.
\end{example}

\begin{solution}
Vectors on the plane:
$\overrightarrow{AB} = \langle 4, -2, 0 \rangle$ and $\overrightarrow{AC} = \langle 7, -5, 1 \rangle$.
The normal vector is their cross product:
\[ \mathbf{n} = \overrightarrow{AB} \times \overrightarrow{AC} = \langle -2, -4, -6 \rangle. \]
The area of $\triangle ABC$ is half the magnitude of the cross product:
\[ \text{Area} = \frac{1}{2} |\overrightarrow{AB} \times \overrightarrow{AC}| = \frac{1}{2} \sqrt{(-2)^2 + (-4)^2 + (-6)^2} = \sqrt{14}. \]
\end{solution}

\subsection{Lines and Planes}

\subsubsection{Parametric Equations of Lines}
A line $L$ through $P_0(x_0, y_0, z_0)$ parallel to $\mathbf{v} = \langle v_1, v_2, v_3 \rangle$ is described by:
\[ \mathbf{r}(t) = \mathbf{r}_0 + t\mathbf{v} \implies 
\begin{cases} 
x = x_0 + tv_1 \\
y = y_0 + tv_2 \\
z = z_0 + tv_3 
\end{cases}
\]

\begin{example}[Example 1.3]
Find the parametric equation of the line passing through $A(3, -2, 0)$ and $B(1, 0, 1)$.
\end{example}

\begin{solution}
Direction vector $\mathbf{v} = \overrightarrow{AB} = \langle -2, 2, 1 \rangle$. Using $A$ as the base point:
\[ \mathbf{r}(t) = \langle 3, -2, 0 \rangle + t\langle -2, 2, 1 \rangle \implies x = 3-2t, y = -2+2t, z = t. \]
\end{solution}

\subsubsection{Equations of Planes}
A plane through $P_0(x_0, y_0, z_0)$ with normal $\mathbf{n} = \langle A, B, C \rangle$ satisfies $\mathbf{n} \cdot \overrightarrow{P_0 P} = 0$:
\[ A(x - x_0) + B(y - y_0) + C(z - z_0) = 0 \quad \text{or} \quad Ax + By + Cz = D. \]

\begin{example}[Example 1.4]
Find the equation of the plane passing through $A(0, 2, -1)$, $B(4, 0, -1)$, $C(7, -3, 0)$.
\end{example}

\begin{solution}
From Example 1.2, $\mathbf{n} = \langle 1, 2, 3 \rangle$ (simplified from $\langle -2, -4, -6 \rangle$).
Using point $A$:
\[ 1(x - 0) + 2(y - 2) + 3(z - (-1)) = 0 \implies x + 2y + 3z = 1. \]
\end{solution}

\subsection{Parametric Curves}

A parametric curve is defined by $\mathbf{r}(t) = f(t)\mathbf{i} + g(t)\mathbf{j} + h(t)\mathbf{k}$.
\begin{itemize}
    \item \textbf{Velocity:} $\mathbf{v}(t) = \mathbf{r}'(t)$. Tangent to the curve.
    \item \textbf{Speed:} $|\mathbf{v}(t)| = |\mathbf{r}'(t)|$.
    \item \textbf{Acceleration:} $\mathbf{a}(t) = \mathbf{r}''(t)$.
\end{itemize}

\begin{theorem}[Conservation of Angular Momentum]
Let $\mathbf{L}(t) = \mathbf{r}(t) \times m\mathbf{r}'(t)$. If $\mathbf{L}(t) = \mathbf{C}$ (constant vector), then the motion lies in a plane.
\end{theorem}

\begin{proof}
Since $\mathbf{L}$ is a cross product involving $\mathbf{r}$, $\mathbf{L} \cdot \mathbf{r} = 0$. If $\mathbf{L} = \langle A, B, C \rangle$ is constant, then $Ax(t) + By(t) + Cz(t) = 0$, which is the equation of a plane through the origin.
\end{proof}

\begin{property}[Differentiation Rules]
\begin{enumerate}
    \item $\frac{d}{dt} [f(t)\mathbf{u}(t)] = f'(t)\mathbf{u}(t) + f(t)\mathbf{u}'(t)$
    \item $\frac{d}{dt} [\mathbf{u}(t) \cdot \mathbf{v}(t)] = \mathbf{u}'(t) \cdot \mathbf{v}(t) + \mathbf{u}(t) \cdot \mathbf{v}'(t)$
    \item $\frac{d}{dt} [\mathbf{u}(t) \times \mathbf{v}(t)] = \mathbf{u}'(t) \times \mathbf{v}(t) + \mathbf{u}(t) \times \mathbf{v}'(t)$
\end{enumerate}
\end{property}

\begin{example}[Example 1.6]
Show that if a particle travels at a uniform speed $C$, its velocity and acceleration are orthogonal.
\end{example}

\begin{solution}
Given $|\mathbf{r}'(t)| = C$, then $|\mathbf{r}'(t)|^2 = \mathbf{r}'(t) \cdot \mathbf{r}'(t) = C^2$.
Differentiating both sides:
\[ \frac{d}{dt}(\mathbf{r}'(t) \cdot \mathbf{r}'(t)) = 0 \implies \mathbf{r}''(t) \cdot \mathbf{r}'(t) + \mathbf{r}'(t) \cdot \mathbf{r}''(t) = 0 \implies 2\mathbf{r}'(t) \cdot \mathbf{r}''(t) = 0. \]
Thus $\mathbf{v} \cdot \mathbf{a} = 0$, so they are orthogonal.
\end{solution}

\subsection{Arc Length and Curvature}

\begin{theorem}[Arc Length]
The arc length of $\mathbf{r}(t)$ from $t=a$ to $t=b$ is:
\[ L = \int_a^b |\mathbf{r}'(t)| \, dt. \]
\end{theorem}

\begin{example}[Example 1.7]
Find the arc length of $\mathbf{r}(t) = \frac{1}{2}t^2 \mathbf{i} + \frac{2\sqrt{2}}{3}t^{3/2} \mathbf{j} + t \mathbf{k}$ from origin to $(2, 8/3, 2)$.
\end{example}

\begin{solution}
Endpoints correspond to $t=0$ and $t=2$.
$\mathbf{r}'(t) = \langle t, \sqrt{2}t^{1/2}, 1 \rangle$.
\[ |\mathbf{r}'(t)|^2 = t^2 + 2t + 1 = (t+1)^2 \implies |\mathbf{r}'(t)| = t+1. \]
\[ L = \int_0^2 (t+1) dt = \left[ \frac{t^2}{2} + t \right]_0^2 = 2 + 2 = 4. \quad (\text{Note: Text says 10, check calculation: } 2^2/2+2 = 4. \text{Wait, text says} \int (t+1) dt = [t^2/2 + t]_0^2 = 2+2=4. \text{Text Example 1.7 result says 10? } 4+2=6? \text{Let's re-read text. Text says } \int_0^2 (t+1)dt = [t^2/2+t]_0^2 = 10? \text{ No, } 2^2/2+2 = 4. \text{Wait, did text integrate different function? } \int (t^2+...) \text{ No. } t=2 \implies 2+2=4. \text{Wait, if result is 10, maybe bounds are different? Ah, usually these examples use } t^2/2 \text{ somewhere. Let's trust calculation } 4 \text{ unless text has typo or I misread image. Text image says } =10. \text{ Calculation in text image: } t^2/2 + t |_0^2. \text{ If } t=4, 16/2+4 = 12. \text{ If } t=2, 2+2=4. \text{ I will assume calculation } 4 \text{ is correct mathematically, but if you want to copy text exactly, copy text. Let's stick to math logic: } 4). \]
\textit{*Note: Based on standard integration $\int_0^2 (t+1)dt = 4$. If the notes say 10, there might be a typo in the notes' limits or function.*}
\end{solution}

\subsubsection{Arc-Length Parametrization}
A curve $\mathbf{r}(s)$ is arc-length parametrized if $|\mathbf{r}'(s)| = 1$.
\textbf{Procedure to find $\mathbf{r}(s)$:}
\begin{enumerate}
    \item Compute arc length function $s(t) = \int_0^t |\mathbf{r}'(\tau)| d\
